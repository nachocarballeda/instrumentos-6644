\documentclass[a4paper,12pt,twoside]{article}
\usepackage[spanish]{babel}
\usepackage[utf8]{inputenc}
\usepackage{graphicx} %para insertar graficos/imagenes
\usepackage{amsmath} %para escribir matrices
\usepackage{amssymb}
\usepackage{upgreek}
\usepackage{amsfonts} %para poner \mathbb
\usepackage{float} %me deja usar la H de 'here' en los graficos para ponerlos donde yo quiera
\usepackage{anysize} %me permite definir los margenes como quiera
\usepackage{multirow} %para tablas con multicolumna
\usepackage{fancyhdr} % activamos el paquete
\usepackage{dcolumn}
\usepackage{multirow}
\usepackage{caption, subcaption}
\newcommand{\quotes}[1]{``#1''}
\captionsetup[table]{textfont=it,position=top}
\usepackage{booktabs}

\usepackage[output-decimal-marker={,}]{siunitx} %Para las unidades. Tengo la versión 2

\usepackage[font=footnotesize, labelfont=bf, margin=2.2cm]{caption}

\usepackage{fixme}
\fxsetup{
    status=draft,
    author=,
    layout=inline, % also try footnote or pdfnote
}

\usepackage[T1]{fontenc}

\newcommand{\grad}{$^\circ$}
\newcommand{\codigoMateria}{66.44}
\newcommand{\nombreMateria}{Instrumentos Electrónicos}
\newcommand{\nroTP}{1}
\newcommand{\descripcionTP}{Analizador de Espectro - Ruido}
\newcommand{\tituloTP}{Analizador de espectros - Ruido}
\newcommand{\facultad}{Facultad de Ingeniería}
\newcommand{\universidad}{Universidad de Buenos Aires}
\newcommand{\docentes}{Enrique Zothner}

\pagestyle{fancy} % seleccionamos un estilo
\fancyhead{}
\fancyfoot{}
\lhead{\nombreMateria \, (\codigoMateria)} % texto izquierda de la cabecera
\rhead{\facultad} % texto centro de la cabecera
\cfoot{\thepage}

\marginsize{2cm}{2cm}{1cm}{1.5cm} %izquierda, derecha, arriba, abajo

\newcommand{\Direcrotio}{./}
\newcommand{\HRule}{\rule{\linewidth}{1mm}}

\graphicspath{{./img/}}

\newenvironment{items}{
\begin{itemize}
  \renewcommand{\labelitemi}{$\bullet$}
  \setlength{\itemsep}{3pt}
  \setlength{\parskip}{1pt}
  \setlength{\parsep}{1pt}
}{
\end{itemize}}



\begin{document}

\include{caratula}

\newpage

\section{Sensibilidad y Ruido}
\subsection{Sensibilidad}
El piso de ruido generado por el analizador de espectros limita las mediciones de bajo nivel. El ruido es amplificado por las multiples etapas de amplificación del analizador. En un analizador de espectro este piso de ruido se conoce como DANL (displayed average noise level). 

La potencia del ruido es una funcion del ancho de banda $f(BW)$. Usando una terminacion de 50 ohms en la entrada para prevenir que entren señales externas no de seadas al analizador, veremos que esta terminacion genera una pequeña cantidad de ruido igual a kTB , k : cte Boltzman, T : Temperatura y B : Ancho de banda medido en Hz. De esta forma podemos determinal el DANL. \newline

Como la potencia de ruido es una funcion del ancho de banda este valor es usualmente normalizado a 1 Hz. Entonces a temperatura ambiente esto da $-174 dBm/Hz$

La resolucion del ancho de banda RBW también afecta la medición de la siguiente forma:\newline

\begin{equation}
10*\log{BW_{2}/BW_{1}}
\label{BW}
\end{equation} 

 donde $BW_1$ es la resolucion de comienzo y $BW_2$ la de fin.\newline

Entonces si cambiamos la resolucion por un factor de 10, el piso de ruido mostrado cambia por un factor de 10 dB. En resumen, para señales de banda angosta obtenemos una mejor sensivilidad seleccionando el minimo ancho de banda de resolución. Lo malo de esto es que el barrido es mucho mas lento.

\subsection{Figura de ruido}
La figura de ruido es definida como la degradacion de la señal ruido por pasar a travez del dispositivo. Podemos expresar esta figura como:
\begin{equation*}
F=\frac{S_i/N_i}{S_o/N_o}
\end{equation*}

Donde F = figura de ruido o factor de ruido, $S_i$ = potencia de la señal de entrada, $N_i$ = verdadera potencia de ruido en la entrada, $S_o$ = Potencia de la señal de salida, $N_o$ = potencia de ruido a la salida. Esta expresión se puede simplificar, ya que la señal de salida es la de la entrada multiplicada por la ganancia del analizador y la ganancia de nuestro analizador es igual a la unidad (igual en la entrada que en la salida). Entonces la expresion queda así:

\begin{equation*}
F=N_o/N_i
\end{equation*}

Entonces lo unico que tenemos que hacer para obtener la figura de ruido es comparar el nivel de ruido mostrado en el display con el verdadero en el conector de entrada. Expresando la figura de ruido en dB obtenemos $NF=10*\log{F}=10*\log{N_o}-10*\log{N_i}$

Entonces todo lo que tenemos que hacer para determinar la figura de ruido es medir la potencia de ruido en algun ancho de banda, calular la potencia de ruido que se habria medido a 1 Hz usnado la formula \ref{BW} y comparar el resultado con -174 dBm.

Por ejemplo, si medimos -110 dBm con BW=10 KHz, obtendriamos : $NF=[ruido medido] - 10\log{RBW/1}- kTB_{b=1 Hz} = 24 dB$ 

Como la figura de ruido es independiente del BW podriamos haber seleccionado otro BW y el resultado seria el mismo.
 
Los 24 dB obtenidos en nuestro ejemplo nos dice que una señal sinusoidal debe estar 24 dB por encima de kTB para ser igual al piso de ruido promediado  mostrado en este analizador. Entonces podesmo usar NF para determinar el DANL para un dado BW.

\subsection{Ruido estudiado como una señal}

Cuando querramos usar el analizador de espectros para medir ruido debemos tener en cuenta que por la naturaleza del mismo, este, indica un nivel que es mas bajo que el ruido que estamos midiendo. A continuacion vamos a ver porque sucede esto y como hacer para corregir este efecto.

Entendemos como ruido aleatorio una señal que tiene una amplitud instantanea distribuida de forma Gausiana (como en la figura \ref{ruido_gausiano}). Un ejemplo de esto es el ruido termico Johnson. Una señal asi no tiene componentes discretos en el espectro, por lo que no podemos tomar la amplitud de punto en particular para medir.
Debemos definir a que nos referimos cuando hablamos de \quotes{potencia de la señal}.

Si muestreamos una señal en un instante arbitrario de tiempo podriamos (en teoria) obtener su valor de amplitud. Pero necesitamos una medicion que exprese el nivel promedio de ruido a travez del tiempo. La potencia RMS satisface este requerimiento.

El filtro de video, y el promediador de video reducen las fluctuaciones pico a pico y nos pueden dar un valor estable. Debemos comparar este valor con la potencia o la tensión RMS. El valor RMS de una distribución Gausiana equivale a su desvio estandart $\upvarsigma$.

Vamos a comenzar como el analizador en modo display lineal. El ruido Gausiano es limitado en banda al pasar por la cadena de IF, y su envolvente adquiere una distribucion Rayleigh como en la figura \ref{ruido_ray}. El ruido que vemos en el display del analizador es la salida del detector de envolvente, es una envolvente distribuida como una Rayleigh de la señal de ruido de entrada. Para obtener un valor constante, un valor promedio, usamos el filtro de video o promediador. El valor medio de una distribucion Rayleigh es $1.253 \upvarsigma$. 

\begin{figure}[H]
    %\centering
    \includegraphics[width=0.8\textwidth]{../img/ruido_gausiano.png}
        \caption{El ruido aleatorio tiene una distribución Gausiana}
        \label{ruido_gausiano}
\end{figure}

\begin{figure}[H]
    %\centering
    \includegraphics[width=0.7\textwidth]{../img/ruido_ray.png}
        \caption{La envolvente de una señal de ruido Gaussiano tiene una distribucion Rayleigh}
        \label{ruido_ray}
\end{figure}

De todas formas nuestro analizador es un voltimetro calibrado que responde a picos para indicar el valor rms de una señal senoidal. Para convertir de calores pico a RMS, nuestro analizador escala su salida por $0.707$ (-3 dB). El valor promedio del ruido distribuido como Rayleigh es escalado entonces por el mismo factor, dandonos una lectura de $0.886 \upvarsigma$ ($1.05 dB$ por debajo de $\upvarsigma$). Para equiparar el valor medio mostrado por el analizador a tension RMS de la señal ruido de entrada debemos tener en cuenta el error introducido en el valor mostrado. Notar que este error no es ambiguo, este es un error constante que puede ser corregido agregando 1.05 dB al valor mostrado en el display.\newline

En la mayoria de los analizadores de espectro la escala del display controla la escala en la cual la distribucion de ruido es promedida tanto con el filtro de VBW o con la traza promediadora. Normalmente, usamos el display del analizador es modo logaritmico, y este modo se agrega al error en nuesta medicion de ruido.\newline

La ganancia de un amplificador logaritmico esta en funcion de la amplitud de la señal, los valores altos de ruido no son tan amplificados tanto como los valores pequeños. Como resutado, la salida del detector de envolventes es una distribucion Rayleigh sesgada, y el valor medio que obtenemos del filtro de video o promediado es otro que esta 1.45 dB mas abajo. En el modo logaritmico, entonces, la media o ruido promedio es mostrado 2.5 dB mas abajo.
Otra vez, este error no es ambiguo, y por lo tanto lo podemos corregir.\newline

Otro factor que afecta a las mediciones de ruido es el ancho de banda en el cual la medición es hecha. Un cambio en la resolución del ancho de banda (RBW) afecta el nivel mostrado del ruido interno generado por el analizador. Este RBW afecta en igual medida a las señales de ruido externas. Para comparar las mediciones hechas con diferentes analizadores, debemos conocer los anchos de banda usadod en cada caso.\newline

No solo los 2 o 6 dB del ancho de banda del analizador afectan la medida del nivel de ruido, la forma del filtro  de resolución tambien juega un papel importante.\newline

Para hacer posible las comparaciones, definimos un \quotes{ancho de banda de potencia del ruido}: el ancho de un filtro rectangular que permite pasar la misma potencia de ruido que el filtro del  analizador, ver figura \ref{modelo}. Para un filtro cuasi-Gausiano en un analizador Agilent, el equivalente \quotes{ancho de banda de la potencia del rudo} es de 1.05 a 1.13 veces los 3 dB del ancho de banda, dependiendo de la selectividad del ancho de banda. Por ejemplo para un filtro RBW de 10KHz se tiene una \quotes{ancho de banda de potencia del ruido} en el rango de 10.5 a 11.3 KHz.\newline

\begin{figure}[H]
    %\centering
    \includegraphics[width=0.7\textwidth]{../img/modelado.png}
        \caption{Ganancia de potencia vs frecuencia de un filtro RBW puede ser modelado como un filtro rectangular con el mismo area y nivel pico y un ancho de el \quotes{ancho de banda de potencia de ruido} equivalente}
        \label{modelo}
\end{figure}

Si usamos $10*\log(BW_{2}/BW_{1})$ para ajustar el nivel de ruido mostrado a lo que habriamos medido con \quotes{ancho de banda de potencia del ruido} del mismo valor que con nuestro ancho de banda de 3 dB, encontramos que los ajustes varian de: $10*log(10000/10500) = -0.21dB$ a $10*log(10000/11300)=-0.53 dB$. En otras palabras, si restamos algo entre 0.21 y 0.53 dB del nivel de ruido indicado, tendremos el \quotes{ancho de banda de potencia de ruido} que es conveniente para realizar los calculos. Para los siguientes ejemplos vamos a usar 0.5 dB como un compromiso razonable para la correccion del ancho de banda.
\newline

Vamos a considerar los factores de correccion para calcular el total de correcciones para cada modo de promediado:
\newline

\bold{Promediado Lineal}\newline
Distribución Reyleigh (modo lineal): $1.05$ dB \newline
3-dB/Ruido potencia ancho de banda:  $-0.50$ dB \newline
Correccion total: $0.55$ dB
\newline

\bold{Promediado Logaritmico}\newline
Distribución Reyleigh (modo logaritmico): $2.50$ dB \newline
3-dB/Ruido potencia ancho de banda:  $-0.50$ dB \newline
Correccion total: $2.00$ dB
\newline
  
\bold{Promediado de potencia (tensión rms)}\newline
Distribución potencia: $0.00$ dB \newline
3-dB/Ruido potencia ancho de banda:  $-0.50$ dB \newline
Correccion total: $-0.50$ dB

\end{document}
